\documentclass{beamer}

\usetheme{metropolis}

% babel settings. 
% The babel package modifies font, 
% hyphenation rules, directionality 
% and more, according to the selecet language.

\usepackage[
	bidi=basic, % this option cahnges the directionality 
				% and language settings without the need 
				% of \selectlanguage, or \foreignlanguage, (at least not all the time) 
				% by detecting the glyphs unicode, and applying 
				% unicode bidi algorithm. esentially it implemented
				% unicode bidi algorithm ith Lua, and embeded it into 
				% TeX via Lua callbacks. 
	%
	layout=tabular graphics, % layout=tabualr fix the tabular direction problem that Juda has meantioned.
							 % layout=graphics fix the use of tikz and similar environments, so there is no 
							 % need for \bodydir TLT before such environment anymore.
	%
	hebrew,provide=*		 % the hebrew option et hebrew as the main language, provide=* 
							 % makes babel to load the information for hebrew from the new .ini file
							 % instead of the old .ldf file, equivalent to \babelprovide[import]{hebrew}....
	]{babel}
	
\babelprovide[
	import, % using .ini file instead of .ldf
	%
	onchar=ids fonts % The `onchar` option means that when 
					 % TeX will encounter a glyph with this
					 % language script (latin in this case),
					 % it will change to this language automatically.
					 % Usually when hebrew is the main language one
					 % should add the `letters` option, which will prevent
					 % non-letters (characters with catagory codes 
					 % different from 11) to replace language. 
					 % But I want to replace `=` and `\` in this document, 
					 % since it is used only in english context. We will
					 % fix numbers and other glyphs in the next few lines. 
	]{english}

% \babelcharproperty {< character code >} {direction} {< bidi unicode algorithm directionallity catagory >} 

\babelcharproperty{`\\}{direction}{l} % this will make make the glyph `\` to behave as a strong left character.
									  % It is convinient with verbatim text, e.g., `\macroname`. 


% \babelcharproperty {< character code >} [<last character code>] {locale} {< lang\empty >}  
									  
\babelcharproperty{32}[64]{locale}{}   % These three lines remove all non latin letters ascii characters  
\babelcharproperty{91}[96]{locale}{}   % from the list of glyphs affected by the `onchar` option used for 
\babelcharproperty{123}[126]{locale}{} % the english language. It is not necessary if the `letters` option  
									   % is used. 
									 

\babelcharproperty{`=}{locale}{english}  % Restoring `=` and `\` to the list 
\babelcharproperty{`\\}{locale}{english} % used by the `onchar` option. 


% \babelfont [< lang >] {< family (rm/sf/tt) >} [< font options >] {< font name >}

\babelfont[hebrew]{sf}[RawFeature=+shuffle]{Open Sans Hebrew} % Hebrew quotes are of the form ”מרכאות„ but 
															  % I'm use to write quotes in TeX as ''מרכאות``.
															  % Since luaotfload enables creating font features 
															  % ``on the fly'', I replaced `“` with `„` when the 
															  % hebrew font is used. I also replaced the normal dash
															  % `-` with the hebrew maqaf, since it is more convinient 
															  % to type the normal dash (beacuse it is on the keyboard).
														
% I'm adding colors to the english font just to demonstrate 
% how the latin letters, `=` and `\` will typeset in the 
% correct font automatically.  
 
\babelfont[english]{sf}[Color=blue]{Fira Sans}
\babelfont[english]{tt}[Color=blue,RawFeature={fallback=hebfallback}]{Fira Mono}

% `RawFeature={fallback=hebfallback}` = As you wanted to use the defualt fonts for english, 
% I chose "Fira Mono" for english typewriter, but in the document there is a paragraph which 
% use this font and has hebrew letters (which are not present in this dont), so we can solve 
% it by creating a fallback font. A font to be used when Fira Mono is used and we encounter a 
% glyph that does not exists in the Fira Mono. I chose Miriam Mono CLM, and added the blue 
% as well. 


% Creating new font feature for the hebrew font 
\directlua
{
	fonts.handlers.otf.addfeature
	{
		name = "shuffle",
		type = "multiple",
		data =
		{
			["־"] = {"-"},
			["-"] = {"־"},
			["“"] = {"„"}
		},
	}
	
	% Creating the fallback font. 
	luaotfload.add_fallback
	  (
		"hebfallback",
		{
			"MiriamMonoCLM:mode=harf;script=hebr;color=0000FF;"
		}
	  )
}

% Making the progress bar progress from right to left.
% We don't need `\bodydir TLT` anymore, but the progres 
% direction is still wrong.

\makeatletter
\setbeamertemplate{progress bar in section page}{
	\setlength{\metropolis@progressonsectionpage}{
		\textwidth * \ratio{\insertframenumber pt}{\inserttotalframenumber pt}
	}%
	\begin{tikzpicture}
		\fill[bg] (0,0) rectangle (\textwidth, \metropolis@progressonsectionpage@linewidth);
		\fill[fg] (\textwidth,0) rectangle (\textwidth-\metropolis@progressonsectionpage, \metropolis@progressonsectionpage@linewidth);
	\end{tikzpicture}%
}
\makeatother

%%%%%%%%%%%%%%%%%%%%%%%%%%%%%%%%%%%%%%%%%%%%%%%%%%%%%%%%%%%%%%%%%%%%%%%%%%%
% The following content was written by Júda Ronén and is written here
% to demonstrate the changes in babel over the years. 
% Its is recomended to read the original code from
%  https://gitlab.com/rwmpelstilzchen/beamer-hebrew/blob/master/beamer-hebrew.tex 
%%%%%%%%%%%%%%%%%%%%%%%%%%%%%%%%%%%%%%%%%%%%%%%%%%%%%%%%%%%%%%%%%%%%%%%%%%%

\newcommand*\smallurl[1]{{\scriptsize\url{#1}}}

\author{
	\texorpdfstring{
		יודה רונן
		\quad \smallurl{<https://me.digitalwords.net/>}}{Yuda Ronen}}
\begin{document}
	\title{מצגות Beamer בעברית הן בגדר האפשר}
	\subtitle{החלום שכולנו חיכינו לו מתגשם סוף כל סוף}
	
	\institute{האגודה הבינלאומית ל\TeX\/ נולוגיה עברית}
	\date{אוגוסט 2018}
	\frame{\titlepage}
	
	\begin{frame}{שלום עולם!}
		
		מה יש לנו כאן?
		
		\begin{itemize}
			\item פתרון מודרני שלם (אך לא מושלם) ליצירת מצגות עבריות ב-Beamer.
			\item משתמש במחלקות וחבילות הרגילות, ולא דורש תחזוק של פוֹרְקִים עם התאמה לעברית. למעשה, פחות או יותר הכל עובד מהקופסה חוץ מכמה תיקונים קטנים שנדרשים.
			\item המנוע הוא Lua\TeX\/, על הכיווניות, והפונטים אחראית החבילה Babel, והעיצוב הוא של הערכה Metropolis האלגנטית, הברורה והנקיה.
		\end{itemize}
	\end{frame}
	
	
	\begin{frame}{על כתפי ענקים}
		ככל הידוע לי, היו לפחות שלושה פתרונות קודמים (הקליקו על השמות לקישורים):
		\begin{itemize}
			\item של \href{https://web.archive.org/web/20111206041733/http://www.technion.ac.il/~ronen/latex/beamer_hebrew.html}{רונן אברבנאל}
			\item של \href{https://web.archive.org/web/20150327024158/http://technion.ac.il/~gai/beamer/index.html}{גיא שקד}
			\item של \href{https://tug.org/pipermail/xetex/2009-July/013721.html}{ופא ח׳ליקי}
		\end{itemize}
		
		כמו שאפשר לראות, שני הראשונים כבר לא נמצאים יותר ברשת באופן עצמאי אלא רק בארכיון האינטרנט.
		הפתרון של רונן לא מתקמפל היטב בגרסאות חדשות;
		של גיא ושל ופא מבוססים על מחלקה לא סטנדרטית.
	\end{frame}
	
	
	\begin{frame}{האקים א׳: קווים מפרידים}
		בגדול הכל עובד ישר מהקופסה. עם זאת, יש כמה דברים קטנים שצריך לשנות, כמו שאפשר לראות בקוד המקור של הקובץ הזה.
		
		שניים מהם קשורים לקווים המפרידים בשקופיות מיוחדות:
		
		\begin{itemize}
			\item תיקון של הקו המפריד בכותרת.
			\item תיקון של הקו המפריד בסעיפים והחלפת הכיוון שלו כך שיהיה הגיוני (התקדמות מימין לשמאל).
		\end{itemize}
	\end{frame}
	
	
	\begin{frame}{האקים ב׳: ענייני פונטים}
		ושניים קשורים לפונטים:
		
		\begin{itemize}
			\item משום מה מנגנון בחירת הפונטים של Babel לא עובד טוב, אז צריך לעקוף אותו ולהשתמש ב- fontspec ישירות.
			\item ניקוד מוצג נכון רק אם אין מלל לועזי לפניו בפסקה%
			\footnote{אני משער שהפונט העברי חוזר ללא המאפיין  \texttt{Script=Hebrew}, ההכרחי לניקוד.}.
			כדי לעקוף את הבעיה אפשר לשים את המילה המנוקדת בסביבה מבודדת%
			\footnote{לדוגמה, בתוך \texttt{\textbackslash mbox} בתוך מתמטיקה; לא אלגנטי אבל עובד. אפשר, כמובן, ליצור פקודה יפה שעושה את זה.}.
			
			\begin{itemize}
				\item ככה זה נראה נכון: abc נִקּוּד\\
				\item וככה לא נכון: abc נִקּוּד
			\end{itemize}
		\end{itemize}
	\end{frame}
	
	
	\begin{frame}[fragile]{האקים ג׳: טבלאות}
		לא בדיוק האק, אלא משהו לשים לב אליו: סדר העמודות בטבלאות הוא משמאל לימין ולא מימין לשמאל.
		
		\vfill
		
		\leavevmode\vtop{\hsize=.45\textwidth
			קלט:\\
			\selectlanguage{english}
			\begin{verbatim}
				\begin{tabular}{r|r|r}
					שמאל & אמצע & ימין
				\end{tabular}
			\end{verbatim}
		}
		\hskip .15\textwidth
		\vtop{\hsize=.35\textwidth
			פלט:\\[3mm]
			\begin{tabular}{r|r|r}
				שמאל &
				אמצע &
				ימין
		\end{tabular}}
	\end{frame}
	
	
	\begin{frame}{מה לא עובד?}
		הבעיה היחידה שנתקלתי בה ולא מצאתי פתרון עבורה היא בביבליוגרפיה עם Bib\LaTeX\/. רוב הרשומות ב-\texttt{\textbackslash printbibliography}  מסתדרות מעולה, ושומרות על כיווניות בלי בעיה. לעומת זאת, רשומות של תרגומים שמשתמשות בשדה \texttt{related} ובשדה \texttt{relatedtype} עם הערך \texttt{translationof} מתחרבשות בחלק שמציג את הספר המקורי: כל האותיות מופיעות בסדר הפוך. אם מצאתם פתרון לבעיה, אשמח לשמוע מתחרבשים בחלק שמציג את המקור: כל האותיות מופיעות בסדר הפוך. אם מצאתם פתרון לבעיה, אשמח לשמוע :-) (פרטי יצירת קשר יש בעמוד השער).
		
		דוגמה לבעיה אפשר לראות בתרגום הוולשי של הארי פוטר במצגת הזאת:
		\smallurl{https://ac.digitalwords.net/\#nonacademic-2018-08-23}\\
		קובץ ה-Bib\LaTeX זמין כאן:
		\smallurl{https://gitlab.com/rwmpelstilzchen/bibliography.bib}
	\end{frame}
	
	
	\begin{frame}{מה צריך עוד לעשות?}
		כאמור, אני נתקלתי רק בבעיה אחת שלא הצלחתי לפתור. יכול להיות שאנשים עם צרכים אחרים משלי יתקלו בבעיות אחרות. אם נתקלת בבעיה כללית, אל תהסס/י לפנות אלי: אם מצאת פתרון, אשמח לשמוע אותו; אם לא מצאת פתרון, אשמח לנסות לעזור.
	\end{frame}
	
	
	\begin{frame}{דוגמה למצגת מורכבת}
		הצורך ביצירת מצגת עברית עלה אצלי עבור הרצאה שהכנתי לכנס מיתופיה 2018 (``הארי פוטר וכינויי הגוף השני''), הרצאה שהיא גלגול של הרצאה שהצגתי בכנסים אקדמיים (בלשניים וקלטולוגיים) לקהל הפוטריסטי. לא יכולתי להסתמך על מצגת באנגלית בגלל שלא כולם בקהל קוראים באנגלית באופן שוטף, כך שהייתי צריך לתרגם את המצגת הקיימת שהיתה לי לעברית. PowerPoint של מיקרוסופט או Impress של ליברה-אופיס הן לא אופציות: אני מעדיף למות תריסר מיתות משונות לפני שאגע בדברים האלה. מתוך הצורך הגיעה נבירה ב-\url{tex.stackexchange.com} ויצירת הפתרון הזה.
		
		למה אני מספר את זה? כי אם אתם רוצים לראות מצגת די מורכבת שעובדת יפה עם עברית ושקוד המקור שלה זמין, גשו אל:\\
		\smallurl{https://ac.digitalwords.net/\#nonacademic-2018-08-23}
	\end{frame}
	
	
	
	\part{ככה נראית שקופית של חלק}
	\frame{\partpage}
	
	\section{וככה סעיף לקראת הסוף}
	
	\begin{frame}{הערת-אגב: פונטים למצגות}
		במצגת הזאת משמש הפונט \href{https://en.wikipedia.org/wiki/Open_Sans}{Open Sans} של סטיב מטסון עם \href{https://yaronimus.wordpress.com/2014/01/19/\%D7\%90\%D7\%95\%D7\%A4\%D7\%9F-\%D7\%A1\%D7\%90\%D7\%A0\%D7\%A1-\%D7\%92\%D7\%95\%D7\%A4\%D7\%9F-\%D7\%A8\%D7\%A9\%D7\%AA-\%D7\%97\%D7\%99\%D7\%A0\%D7\%9E\%D7\%99-\%D7\%91\%D7\%A2\%D7\%99\%D7\%A6\%D7\%95\%D7\%91\%D7\%95-\%D7\%A9\%D7\%9C-\%D7\%99\%D7\%90/}{ההרחבה העברית} שעיצב ינק יונטף (``הפונט של הארץ'').
		לדעתי זה פונט מעולה למצגות, בדומה ל-Fira Sans (הפונט הדיפולטיבי של Metropolis); $\mbox{אָלֶף}$ (Alef), עם כל חיבתי אליו, לא מתאים למצגות.
	\end{frame}
\end{document}

